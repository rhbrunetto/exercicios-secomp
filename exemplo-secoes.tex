\documentclass{article}

\usepackage{graphicx}         % Pacote para usar os gráficos
\usepackage[utf8]{inputenc}		% Pacote para usar acentos

\usepackage{lipsum}           % Pacote para ENCHER LINGUIÇA (é bom mostrar pra galera)

\begin{document}

% Fazemos com eles
% Tudo sobrescrevendo os passos. Só separei para irmos nos orientando!!!

% 1)
\section{Minha seção top}
  Textão do Facebook [...]\\
  \lipsum[1] % Mostrando o parágrafo 1 do Lorem Ipsum ...
  



% 2) Vamos adicionar uma nova seção
\section{Minha seção top}
  Textão do Facebook [...]\\
  \lipsum[1] % Mostrando o parágrafo 1 do Lorem Ipsum ...
  
\section{Minha outra seção menos top que a outra}
  Mais textão\\
  \lipsum[2] % Mostrando o parágrafo 2 do Lorem Ipsum ...




% 3) Vamos Trocar a ordem das seções e ver o que acontece com a enumeração delas
\section{Minha outra seção menos top que a outra}
  Textão do Facebook [...]\\
  \lipsum[1] % Mostrando o parágrafo 1 do Lorem Ipsum ...

\section{Minha seção top}
  Mais textão\\
  \lipsum[2] % Mostrando o parágrafo 2 do Lorem Ipsum ...


  % ----------------------------------------------------------------------------- %
  % FINALIZA AQUI O EXEMPLO DAS SEÇÕES. OS PASSOS 4 E 5 SÃO REFERENTES AO SUMÁRIO %
  % ----------------------------------------------------------------------------- %


% 4) Vamos adicionar um sumário nessa bagunça
\tableofcontents 

\section{Minha outra seção menos top que a outra}
  Textão do Facebook [...]\\
  \lipsum[1] % Mostrando o parágrafo 1 do Lorem Ipsum ...

\section{Minha seção top}
  Mais textão\\
  \lipsum[2] % Mostrando o parágrafo 2 do Lorem Ipsum ...



% 5) Vamos deixar o sumário numa página só dele
\tableofcontents
\newpage        % Esse comando cria uma nova página para o conteúdo que está abaixo dele (tipo o quebra de página do Word)

\section{Minha outra seção menos top que a outra}
  Textão do Facebook [...]\\
  \lipsum[1] % Mostrando o parágrafo 1 do Lorem Ipsum ...

\section{Minha seção top}
  Mais textão\\
  \lipsum[2] % Mostrando o parágrafo 2 do Lorem Ipsum ...



\end{document}
