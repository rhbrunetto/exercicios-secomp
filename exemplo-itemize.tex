\documentclass{article}

\usepackage[utf8]{inputenc}		% Pacote para usar acentos


\begin{document}

% Fazemos com eles:
% E vamos incrementando. Tudo na mesma lista.
% Façam devagar com a galera, sem pressa. Essa parte pode ser meio confusa.
% Toda vez que terminarmos um passo, vamos mostrando o resultado

Minha lista:

% 1) DICA VALIOSA: Use identação
\begin{itemize}
	\item Dragon Ball
	\item Naruto
\end{itemize}

% 2) Vamos adicionar um subitem 
\begin{itemize}
  \item Dragon Ball
    \subitem é melhor que
  \item Naruto
\end{itemize}
  
% 3) Vamos trocar o marcador de um item (igual ao enumerate)
% Por padrão, o marcador é um símbolo MATEMÁTICO \bullet
\begin{itemize}
  \item[a] Dragon Ball
    \subitem é melhor que
	\item Naruto
\end{itemize}

% 4) Vamos adicionar um marcador ao subitem
% Note que NÃO precisa dos colchetes []
\begin{itemize}
  \item[a] Dragon Ball
    \subitem $\bullet$ é melhor que     % \bullet é o símbolo padrão, e é MATEMÁTICO (por isso os $$)
	\item Naruto
\end{itemize}

% 4) Vamos apagar os marcadores que fizemos antes e vamos adicionar um marcador novo à TODOS os itens
% Primeiro vamos trocar nosso subitem por um itemize (lembre que podemos aninhar eles)
\begin{itemize}
  \item Dragon Ball
  
    \begin{itemize}
      \item é
      \item melhor
      \item que
    \end{itemize}

	\item Naruto
\end{itemize}
% Mostre como fica.

% Para trocar os marcadores, precisamos REESCREVER o comando de rótulo
\renewcommand\labelitemi{-} % Estamos trocando o \bullet por um traço simples
% Perceba o nome do comando: (escreva no quadro se achar melhor)
  % label (de rótulo)
  % item (de item - hehehe)
  % i (de nível I, ou seja, do primeiro nível de itens)
% Assim, se tivermos itemizes aninhados (um dentro do outro), podemos trocar seus marcadores com
\renewcommand\labelitemii{.} % Estamos colocando marcador ponto (.) nos \item que estão aninhados no nível II

\begin{itemize}
  \item Dragon Ball

  \begin{itemize}
    \item é
    \item melhor
    \item que
  \end{itemize}

	\item Naruto
\end{itemize}


% COMENTÁRIOS IMPORTANTES DE SE FAZER:
%
% ESSA NÃO É A ÚNICA FORMA DE TROCAR TODOS OS MARCADORES
% PODEMOS UTILIZAR O PACOTE enumeritem E FAZER EXATAMENTE COMO FIZEMOS COM O enumerate (VAI FUNCIONAR)
% EXISTEM OUTROS PACOTES QUE TAMBÉM PERMITEM FAZER ESSE TIPO DE COISA. EX: O PACOTE enumerate (sim, ele tem o mesmo nome do ambiente)
% NÃO VAMOS MOSTRAR POR FALTA DE TEMPO

\end{document}
