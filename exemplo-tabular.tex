\documentclass{article}

\usepackage[utf8]{inputenc}		% Pacote para usar acentos


\begin{document}

% Fazemos com eles:
% E vamos incrementando. Tudo na mesma tabela.

% 1)
  \begin{tabular}{c}
    Bom dia
  \end{tabular}

% 2) Agora vamos adicionar outra coluna, só que com alinhamento à esquerda
  \begin{tabular}{cl}
    Bom dia & Boa noite
  \end{tabular}

% 3) Agora vamos desenhar o divisor entre as colunas
  \begin{tabular}{c | l}
    Bom dia & Boa noite
  \end{tabular}

% 4) Agora vamos fazer outras linhas
  \begin{tabular}{c | l}
    Bom dia & Boa noite \\
    Boa tarde & % Coluna deixada em branco (explicar que precisa adicionar o &)
  \end{tabular}

% 5) Agora vamos adicionar uma terceira coluna, com alinhamento à direita
% Precisamos editar as linhas para ter dois & cada uma (pois o & é o divisor da coluna)
  \begin{tabular}{c | l | r}
    Bom dia &  & Boa noite \\
    Boa tarde & Buenas tardes & Buenas notches
  \end{tabular}
  
% 6) Agora vamos desenhar os divisores horizontais (comando \hline)
% Pense que é necessário desenhar um divisor logo após uma nova linha da tabela e ao final da tabela
  \begin{tabular}{c | l | r}
    \hline  % Adiciona o "teto" da tabela
	  Bom dia   &               & Boa noite      \\ \hline % Adiciona divisor
    Boa tarde & Buenas tardes & Buenas notches \\ % Pulamos linha para ir pra baixo
    \hline % Adiciona o "piso" da tabela
\end{tabular}


\end{document}
