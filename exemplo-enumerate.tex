\documentclass{article}

\usepackage[utf8]{inputenc}		% Pacote para usar acentos

% NÃO adicionar ainda, vamos adicionar esse pacote no passo 4)
\usepackage{enumitem}


\begin{document}

% Fazemos com eles:
% E vamos incrementando. Tudo na mesma lista.
% Façam devagar com a galera, sem pressa. Essa parte pode ser meio confusa.

Minha lista:

% 1) DICA VALIOSA: Use identação
\begin{enumerate}
  \item Dragon Ball
  \item Naruto
\end{enumerate}

% 2) Vamos trocar o marcador de um item
\begin{enumerate}
  \item[a)] Dragon Ball
  \item Naruto
\end{enumerate}

% 3) E se eu quisesse colocar uma seta?
\begin{enumerate}
  \item[$\rightarrow$] Dragon Ball      % Isso porque a SETA (comando \rightarrow) é um símbolo do ambiente MATEMÁTICO
  \item Naruto
\end{enumerate}

% 4) E se eu quisesse tudo em ordem alfabética?
% Precisamos de um pacote especial: o pacote enumitem (ADICIONAR NO PREÂMBULO \usepackage{enumitem})
% Esse pacote sobrescreve o ambiente enumerate original e permite adicionar um parâmetro LABEL entre []
% Esse parâmetro label NÃO TEM NADA A VER com o \label que usamos para referenciar coisas!!! É só que os caras têm o mesmo nome!
\begin{enumerate}[label=\alph*] %\alph* é um comando do TeX que gera uma lista contendo o alfabeto (hack estilo python)
  \item Dragon Ball
  \item Naruto
\end{enumerate}

% 5) E se eu quisesse tudo em números romanos?
% Basta trocar a listagem para a lista dos caracteres romanos
\begin{enumerate}[label=\roman*]
  \item Dragon Ball
  \item Naruto
\end{enumerate}

% 6) E se eu quisesse tudo em números romanos ENTRE PARÊNTESES?
% Basta adicionar os parênteses no label
\begin{enumerate}[label=(\roman*)]
  \item Dragon Ball
  \item Naruto
\end{enumerate}

\end{document}
